\documentclass[a4paper]{article}
\usepackage{automultiplechoice}
\begin{document}

%%%%%%%%%%%%%%%%%%%%%%%%%%%%%%%%%%%%%%%%%%%%%%%
% Premier morceau pour corriger un bug...
%%%%%%%%%%%%%%%%%%%%%%%%%%%%%%%%%%%%%%%%%%%%%%%

\makeatletter
\renewcommand{\insertgroup}[2][0]{%
  \AMCtok@max=#1\relax%
  \ifnum\the\AMCtok@max<1%
    \AMCtok@max=\csname #2@k\endcsname%
  \fi%
  \AMCtok@ik=\z@%
  {\loop%
    \advance\AMCtok@ik\@ne\relax%
    {\the\csname #2@\romannumeral\AMCtok@ik\endcsname}%
  \ifnum\AMCtok@ik<\AMCtok@max\repeat}%
}
\makeatother

%%%%%%%%%%%%%%%%%%%%%%%%%%%%%%%%%%%%%%%%%%%%%%%

% graine generateur aleatoire

\AMCrandomseed{1237893}

% les questions du groupe A

\element{gra}{
  \begin{question}{a1}
    Texte de la question A1
    \begin{reponses}
      \bonne{Bonne}
      \mauvaise{Mauvaise}
    \end{reponses}
  \end{question}
}

\element{gra}{
  \begin{question}{a2}
    Texte de la question A2
    \begin{reponses}
      \bonne{Bonne}
      \mauvaise{Mauvaise}
    \end{reponses}
  \end{question}
}

% les questions du groupe B

\element{grb}{
  \begin{question}{b1}
    Texte de la question B1
    \begin{reponses}
      \bonne{Bonne}
      \mauvaise{Mauvaise}
    \end{reponses}
  \end{question}
}

\element{grb}{
  \begin{question}{b2}
    Texte de la question B2
    \begin{reponses}
      \bonne{Bonne}
      \mauvaise{Mauvaise}
    \end{reponses}
  \end{question}
}

\element{grb}{
  \begin{question}{b3}
    Texte de la question B3
    \begin{reponses}
      \bonne{Bonne}
      \mauvaise{Mauvaise}
    \end{reponses}
  \end{question}
}

% le groupe general qui contient deux elements : ceux-ci presentent et
% restitutent chacun des deux groupes A et B

\element{grtous}{
  \begin{center}
    \bf\large Groupe A
  \end{center}
  \restituegroupe{gra}
}

\element{grtous}{
  \begin{center}
    \bf\large Groupe B
  \end{center}
  \restituegroupe{grb}
}

% les copies...

\exemplaire{10}{

  \begin{center}
    \bf\Large Deux groupes m\'elang\'es.
  \end{center}

  \melangegroupe{gra}
  \melangegroupe{grb}
  \melangegroupe{grtous}

  \restituegroupe{grtous}

}

\end{document}
