\documentclass{article}

\usepackage[latin1]{inputenc}
\usepackage[T1]{fontenc}

\usepackage[bloc,completemulti]{automultiplechoice}

\begin{document}

% preparation du groupe de questions appel� qqs (on mettra au maximum
% 20 elements dans ce groupe) :

\nouveaugroupe{qqs}{20}

\element{qqs}{
\begin{question}{le bon choix}
  Combien de points voulez-vous � cette question~?
  \begin{reponses}
    \bonne{Le maximun : 10}\bareme{10}
    \mauvaise{Seulement 5}\bareme{5}
    \mauvaise{Deux me suffiront}\bareme{2}
    \mauvaise{Aucun, merci}\bareme{0}
  \end{reponses}
\end{question}
}

\element{qqs}{
\begin{questionmult}{engrange}
  Engrangez des points gratuitement en cochant les cases ci-dessous~:
  \begin{reponses}
    \bonne{2 points}\bareme{b=2}
    \mauvaise{Un point n�gatif}\bareme{b=0,m=-1}
    \bonne{3 points}\bareme{b=3}
    \bonne{1 point}
    \bonne{Un demi point}\bareme{b=0.5}
  \end{reponses}
\end{questionmult}
}

\element{qqs}{
\begin{questionmult}{2 justes}\bareme{b=3,d=-9,p=0}
  Il faut cocher exactement comme il faut pour avoir trois points, sinon vous n'en
  aurez aucun.
  \begin{reponses}
    \mauvaise{Fausse}
    \mauvaise{Fausse}
    \bonne{Juste}
    \bonne{Juste}
  \end{reponses}
\end{questionmult}
}

\element{qqs}{
\begin{questionmult}{tout}\bareme{d=-3,p=0}
  Deux points pour tout juste, et un point pour chaque erreur...
  \begin{reponses}
    \bonne{Bonne r�ponse}
    \bonne{Ceci est juste}
    \bonne{Exact}
    \mauvaise{Faux~!}
    \mauvaise{Ne pas cocher~!}
  \end{reponses}
\end{questionmult}
}

\element{qqs}{
\begin{question}{attention}\bareme{b=2}
  Alors l�, la r�ponse tr�s fausse m�rite sanction (-2 points), mais
  viser juste rapporte 2 points.
  \begin{reponses}
    \bonne{C'est bon !}
    \mauvaise{Pas bon}
    \mauvaise{Pas bon}
    \mauvaise{Pas bon}
    \mauvaise{Tr�s faux !}\bareme{-2}
  \end{reponses}
\end{question}
}

\element{qqs}{
\begin{questionmult}{au choix}
  Choisissez vos points :
  \begin{reponses}
    \bonne{J'en veux 2}\bareme{b=2}
    \mauvaise{J'en donne trois}\bareme{b=0,m=3}
    \bonne{J'en veux un (et sinon j'en perds un)}\bareme{m=-1}
  \end{reponses}
\end{questionmult}
}

%%%%%%%%%%%%%%%%%%%%%%%%%%%%%%%%%%%%%%%%%%%%%%%%%%%%%%%%%%%%%%%%%%%%%%

\exemplaire{20}{

\noindent{\bf QCM  \hfill TEST DE BAR�ME}

\vspace*{.5cm}
\begin{minipage}{.4\linewidth}
\centering\large\bf Test\\ Examen du 01/01/2008\end{minipage}
\champnom{\fbox{\begin{minipage}{.5\linewidth}
Nom et pr�nom :

\vspace*{.5cm}\dotfill
\vspace*{1mm}
\end{minipage}}}

\begin{center}\em
Dur�e : 10 minutes.
\end{center}
\vspace{1ex}

%%%%%%%%%%%%%%%%%%%%%%%%%%%%%%%%%%%%%%%%%%%%%%%%%%%%%%%%%%%%%%%%%%%%%%

\melangegroupe{qqs}

\restituegroupe{qqs}

%%%%%%%%%%%%%%%%%%%%%%%%%%%%%%%%%%%%%%%%%%%%%%%%%%%%%%%%%%%%%%%%%%%%%%

\cleardoublepage

}

\end{document}