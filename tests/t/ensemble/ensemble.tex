%
% Copyright (C) 2012-2017 Alexis Bienvenue
%
% This file is part of Auto-Multiple-Choice
%
% Auto-Multiple-Choice is free software: you can redistribute it
% and/or modify it under the terms of the GNU General Public License
% as published by the Free Software Foundation, either version 2 of
% the License, or (at your option) any later version.
%
% Auto-Multiple-Choice is distributed in the hope that it will be
% useful, but WITHOUT ANY WARRANTY; without even the implied warranty
% of MERCHANTABILITY or FITNESS FOR A PARTICULAR PURPOSE.  See the GNU
% General Public License for more details.
%
% You should have received a copy of the GNU General Public License
% along with Auto-Multiple-Choice.  If not, see
% <http://www.gnu.org/licenses/>.
%
\documentclass[a4paper]{article}

\usepackage[utf8x]{inputenc}    
\usepackage[T1]{fontenc}

\usepackage[francais,bloc,completemulti,ensemble]{automultiplechoice}    
\begin{document}

\AMCrandomseed{1237893}

\def\AMCformQuestion#1{\vspace{\AMCformVSpace}\par {\sc Question #1 :} }    

\element{general}{
  \begin{question}{prez}    
    Parmi les personnalités suivantes, laquelle a été présidente de la république française~?
    \begin{reponses}
      \bonne{René Coty}
      \mauvaise{Alain Prost}
      \mauvaise{Marcel Proust}
      \mauvaise{Claude Monet}
    \end{reponses}
  \end{question}
}

\element{general}{
  \begin{questionmult}{pref}    
    Parmi les villes suivantes, lesquelles sont des préfectures~?
    \begin{reponses}
      \bonne{Poitiers}
      \mauvaise{Sainte-Menehould}
      \bonne{Avignon}
    \end{reponses}
  \end{questionmult}
}

\element{general}{
  \begin{question}{nb-ue}
    Combien d'états sont membres de l'Union Européenne en janvier 2009 ?
    \begin{reponseshoriz}[o]
      \mauvaise{15}
      \mauvaise{21}
      \mauvaise{25}
      \bonne{27}
      \mauvaise{31}
    \end{reponseshoriz}
  \end{question}
}

\exemplaire{5}{    

%%% debut de l'en-tête des copies :    

\noindent{\bf QCM  \hfill TEST}

\vspace*{.5cm}
\begin{minipage}{.4\linewidth}
  \centering\large\bf Test\\ Examen du 01/01/2008
\end{minipage}

\begin{center}\em
Durée : 10 minutes.

  Aucun document n'est autorisé.
  L'usage de la calculatrice est interdit.

  Les questions faisant apparaître le symbole \multiSymbole{} peuvent
  présenter zéro, une ou plusieurs bonnes réponses. Les autres ont
  une unique bonne réponse.

  Des points négatifs pourront être affectés à de \emph{très
    mauvaises} réponses.
\end{center}
\vspace{1ex}

%%% fin de l'en-tête

\melangegroupe{general}
\restituegroupe{general}

\AMCcleardoublepage    

\AMCdebutFormulaire    

%%% début de l'en-tête de la feuille de réponses

{\large\bf Feuille de réponses :}
\hfill \champnom{\fbox{    
    \begin{minipage}{.5\linewidth}
      Nom et prénom :
      
      \vspace*{.5cm}\dotfill
      \vspace*{1mm}
    \end{minipage}
  }}

\begin{center}
  \bf\em Les réponses aux questions sont à donner exclusivement sur cette feuille :
  les réponses données sur les feuilles précédentes ne seront pas prises en compte.
\end{center}

%%% fin de l'en-tête de la feuille de réponses

\formulaire    

\clearpage    

}  

\end{document}
