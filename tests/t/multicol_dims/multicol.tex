\documentclass[a4paper]{article}

\usepackage[utf8x]{inputenc}    
\usepackage[T1]{fontenc}
\usepackage{ProfCollege}

\usepackage[francais,bloc,completemulti,ensemble]{automultiplechoice}    
\begin{document}

\AMCrandomseed{1237893}

\def\AMCformQuestion#1{\vspace{\AMCformVSpace}\par {\sc Question #1 :} }    

\element{general}{
  \begin{question}{prez}    
    Parmi les personnalités suivantes, laquelle a été présidente de la république française~?
    \begin{reponses}
      \bonne{René Coty}
      \mauvaise{Alain Prost}
      \mauvaise{Marcel Proust}
      \mauvaise{Claude Monet}
    \end{reponses}
  \end{question}
}

\element{general}{
  \begin{question}{nb-ue}
    Combien d'états sont membres de l'Union Européenne en janvier 2009 ?
    \begin{reponseshoriz}[o]
      \mauvaise{15}
      \mauvaise{21}
      \mauvaise{25}
      \bonne{27}
      \mauvaise{31}
    \end{reponseshoriz}
  \end{question}
}

\element{general}{
  \begin{questionmult}{fx}
    On considère la fonction $f:\mathbb{R}\longrightarrow\mathbb{R}$ définie par
    $f(x)=4x+2$. Alors :
    \vspace{1ex}
    
    \MultiCol[t]{0.25/0.70}{
      {\AMCnoCompleteMulti\begin{reponses}
          \bonne{$f(3)=14$}
          \mauvaise{$f(3)=20$}
        \end{reponses}}
      §
      \begin{reponses}
        \mauvaise{On ne peut pas calculer $f(3)$ car 3 est un nombre entier}
      \end{reponses}
    }
  \end{questionmult}
}

\element{general}{
\begin{questionmult}{R.ect-a}\bareme{default.A=0,default.AA=0,default.B=0,default.BB=0,default.C=0,default.CC=0,formula="max(0,A+AA)+max(0,B+BB)+max(0,C+CC)"}
  Un écart type est un indicateur\dots
  \vspace{1ex}
  
  \MultiCol[t]{0.28/0.36/0.36}{
    {\AMCnoCompleteMulti\begin{reponses}
    \mauvaise{de tendance centrale}\bareme{set.A=-1}
    \bonne{de dispersion}\bareme{set.AA=1}
    \end{reponses}}
    §
    {\AMCnoCompleteMulti\begin{reponses}
    \bonne{sensible aux valeurs extrêmes}\bareme{set.BB=1}
    \mauvaise{peu sensible aux extrêmes}\bareme{set.B=-1}
    \end{reponses}}
    §
    \begin{reponses}
    \bonne{de même unité que les valeurs}\bareme{set.CC=1}
    \mauvaise{d'unité différente des valeurs}\bareme{set.C=-1}
    \end{reponses}
  }
\end{questionmult}
}

\element{general}{
\begin{questionmult}{R.ect-b}\bareme{default.A=0,default.AA=0,default.B=0,default.BB=0,default.C=0,default.CC=0,formula="max(0,A+AA)+max(0,B+BB)+max(0,C+CC)"}
  Un écart type est un indicateur\dots
  \AMCnoCompleteMulti

  \emph{(concernant son objectif)}
  \begin{reponses}
    \mauvaise{de tendance centrale}\bareme{set.A=-1}
    \bonne{de dispersion}\bareme{set.AA=1}
  \end{reponses}

  \emph{(concernant sa sensibilité)}
  \begin{reponses}
    \bonne{sensible aux valeurs extrêmes}\bareme{set.BB=1}
    \mauvaise{peu sensible aux valeurs extrêmes}\bareme{set.B=-1}
  \end{reponses}

  \emph{(concernant sa dimension)}
  \begin{reponses}
    \bonne{de même unité que les valeurs}\bareme{set.CC=1}
    \mauvaise{d'unité différente des valeurs}\bareme{set.C=-1}
  \end{reponses}
\end{questionmult}
}

\exemplaire{5}{    

%%% debut de l'en-tête des copies :    

\noindent{\bf QCM  \hfill TEST}

\begin{center}
  \large\bf QUESTIONS
\end{center}
\vspace{1ex}

%%% fin de l'en-tête

\melangegroupe{general}
\restituegroupe{general}

\AMCcleardoublepage    

\AMCdebutFormulaire    

%%% début de l'en-tête de la feuille de réponses

{\large\bf Feuille de réponses :}
\hfill \champnom{\fbox{    
    \begin{minipage}{.5\linewidth}
      Nom et prénom :
      
      \vspace*{.5cm}\dotfill
      \vspace*{1mm}
    \end{minipage}
  }}

\begin{center}
  \bf\em Les réponses aux questions sont à donner exclusivement sur cette feuille :
  les réponses données sur les feuilles précédentes ne seront pas prises en compte.
\end{center}

%%% fin de l'en-tête de la feuille de réponses

\formulaire    

\clearpage    

}  

\end{document}
