%
% Copyright (C) 2016-2021 Alexis Bienvenüe <paamc@passoire.fr>
%
% This file is part of Auto-Multiple-Choice
%
% Auto-Multiple-Choice is free software: you can redistribute it
% and/or modify it under the terms of the GNU General Public License
% as published by the Free Software Foundation, either version 2 of
% the License, or (at your option) any later version.
%
% Auto-Multiple-Choice is distributed in the hope that it will be
% useful, but WITHOUT ANY WARRANTY; without even the implied warranty
% of MERCHANTABILITY or FITNESS FOR A PARTICULAR PURPOSE.  See the GNU
% General Public License for more details.
%
% You should have received a copy of the GNU General Public License
% along with Auto-Multiple-Choice.  If not, see
% <http://www.gnu.org/licenses/>.
%
\documentclass[a4paper]{article}

\usepackage{multicol}
\usepackage[box]{automultiplechoice}    

\usepackage{polyglossia}
\setmainlanguage{english}
\setotherlanguage{arabic}
\newfontfamily{\arabicfont}[Script=Arabic,Scale=1]{Rasheeq}
\RTLmulticolcolumns

\begin{document}

\def\AMCbeginQuestion#1#2{\par\noindent\hrule\vspace{1ex}\par}

\begin{examcopy}   

%%% beginning of the test sheet header:    

\noindent\begin{minipage}[t]{.25\linewidth}
\champnom{\fbox{\begin{Arabic}\begin{minipage}{.9\linewidth}
 ‫الفصل:‬
   \vspace{5ex}
 
   ‫اسم الطالب:‬
   \vspace{5ex}
  \end{minipage}\end{Arabic}}}\end{minipage}
  \hfill
%
\fbox{\begin{minipage}[t]{.7\linewidth}
        \begin{Arabic}
	    	$-$ ‫فضل قم بتظليل السربع الصحيح بالسرسام .‬
	    	
  		    $-$ ‫فضل ل تكتب أي نص على الورقة عدا عن اسك وفصلك ، ول تكتب خارج هذا التستطيل.‬
  		    
            $-$ ‫مع تنيات لك بالتوفيق والنجاح.‬
        \end{Arabic}
      \end{minipage}}

\begin{center}
%\includegraphics{allquations.pdf}

  \hrule\vspace{2mm}
\end{center}
%%% end of the header


\begin{multicols}{3}\columnseprule=.4pt

 \begin{Arabic}\begin{question}{a}
        ‫السؤال الول:البعد بي عقدتي ف الوجات الوقوفة يساوي:‬
    \begin{choices}
      \wrongchoice{ربع طول موجي‬}
      \correctchoice{نصف طول موجي‬}
      \wrongchoice{ةثلةثة أرباع طول موجي‬}
      \wrongchoice{طول موجي واحد‬}
    \end{choices}
  \end{question}\end{Arabic}

  \begin{Arabic}\begin{question}{b}
  ‫السؤال الثان :إذا اصدر وتر طوله 30 سم نغمة مع شوكة ترددها 320 هيتز ، فإن تردد الشوكة الت تصدر نغمة عندما يكون طوله 20 سم يساوي (هيتز):
    \columnseprule=0pt\begin{multicols}{2}
    \begin{choices}[o]
      \wrongchoice{120}
      \wrongchoice{240}
      \correctchoice{480}
      \wrongchoice{512‬}
    \end{choices}
    \end{multicols}
  \end{question}\end{Arabic}

 \begin{Arabic}\begin{question}{c}
       ‫السؤال الثالث:الاهاز الستخدم لدراسة العوامل الؤةثرة ف اهتزاز الوتار يسمى :‬
    \columnseprule=0pt\begin{multicols}{2}
    \begin{choices}
      \correctchoice{الصوات‬}
      \wrongchoice{السفيومتر‬}
      \wrongchoice{الترمومتر}
      \wrongchoice{الفوتوميتر‬}
    \end{choices}
    \end{multicols}
  \end{question}\end{Arabic}

  \begin{Arabic}\begin{question}{d}
  ‫السؤال الرابع :"اضطرار جسم للهتزاز بسبب ملمسته لسم آخر ماهتز" هو تعريف:‬
    \begin{choices}
      \correctchoice{الهتزاز القسري}
      \wrongchoice{الهتزاز الرنين}
      \wrongchoice{الطول الوجي}
      \wrongchoice{التردد‬}
    \end{choices}
  \end{question}\end{Arabic}
  
  
  
 \begin{Arabic}\begin{question}{e}
 ‫السؤال الامس: الرني الثالث ف العمدة الوائية الغلقة يعطي النغمة :
    \columnseprule=0pt\begin{multicols}{2}
    \begin{choices}
      \wrongchoice{الول}
      \correctchoice{الثانية}
      \wrongchoice{الثالثة‬}
      \wrongchoice{الرابعة}
    \end{choices}
    \end{multicols}
  \end{question}\end{Arabic}


  \begin{Arabic}\begin{question}{f}
السؤال السادس : تصحيح الناهاية لعمود هوائي مفتوح نصف قطره 10 سم تساوي (سم):
    \columnseprule=0pt\begin{multicols}{2}
    \begin{choices}[o]
      \wrongchoice{5}
      \correctchoice{6}
      \wrongchoice{10}
      \wrongchoice{100}
    \end{choices}
    \end{multicols}
  \end{question}\end{Arabic}
  
 \begin{Arabic}\begin{question}{g}
السؤال السابع:"كمية الطاقة الضوئية الرئية الساقطة عموديا على وحدة الساحات من السطح ف الثانية" هو تعريف :‬
    \begin{choices}
      \wrongchoice{قوة الضاءة}
      \correctchoice{شدة الستضاءة}
      \wrongchoice{الفوتوميتر‬}
      \wrongchoice{انعكاس الضوء}
    \end{choices}
  \end{question}\end{Arabic}
  


  \begin{Arabic}\begin{question}{h}
السؤال الثامن : إذا وضع مصدر ضوئي قوته 60 شعه عموديا على ارتفاع 2 متر من ورقة ، فإن شدة استضاءة الورقة تساوي (شعة مترية):
    \columnseprule=0pt\begin{multicols}{2}
    \begin{choices}[o]
      \wrongchoice{20}
      \wrongchoice{30}
      \correctchoice{51}
      \wrongchoice{120}
    \end{choices}
    \end{multicols}
  \end{question}\end{Arabic}
  

   \begin{Arabic}\begin{question}{i}
السؤال التاسع : من الصادر غي ذاتية الضاءة :
    \columnseprule=0pt\begin{multicols}{2}
    \begin{choices}
      \wrongchoice{الشمعة}
      \correctchoice{القمر‬}
      \wrongchoice{الشمس}
      \wrongchoice{الصباح‬}
    \end{choices}
    \end{multicols}
  \end{question}\end{Arabic}


\end{multicols}  

\end{examcopy}

\end{document}
