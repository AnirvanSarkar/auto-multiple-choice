%
% Copyright (C) 2023 Alexis Bienvenüe
%
% This file is part of Auto-Multiple-Choice
%
% Auto-Multiple-Choice is free software: you can redistribute it
% and/or modify it under the terms of the GNU General Public License
% as published by the Free Software Foundation, either version 2 of
% the License, or (at your option) any later version.
%
% Auto-Multiple-Choice is distributed in the hope that it will be
% useful, but WITHOUT ANY WARRANTY; without even the implied warranty
% of MERCHANTABILITY or FITNESS FOR A PARTICULAR PURPOSE.  See the GNU
% General Public License for more details.
%
% You should have received a copy of the GNU General Public License
% along with Auto-Multiple-Choice.  If not, see
% <http://www.gnu.org/licenses/>.
%
\documentclass[a4paper,12pt]{article}

\usepackage[utf8]{inputenc}
\usepackage[T1]{fontenc}
\usepackage{automultiplechoice}
\usepackage{fp} 

\begin{document}

\element{Base 10}{
  \begin{questionmultx}{73sd}
    73
    \begin{center}
      \AMCnumericChoices{73}{decimals=0,digits=4,sign=false}
    \end{center}
  \end{questionmultx}
  
  \begin{questionmultx}{73ad}
    73
    \begin{center}
      \AMCnumericChoices{73}{decimals=2,digits=4,sign=false}
    \end{center}
  \end{questionmultx}
  
  \begin{questionmultx}{730sd}
    73.0
    \begin{center}
      \AMCnumericChoices{73.0}{decimals=0,digits=4,sign=false}
    \end{center}
  \end{questionmultx}
  
  \begin{questionmultx}{730ad}
    73.0
    \begin{center}
      \AMCnumericChoices{73.0}{decimals=2,digits=4,sign=false}
    \end{center}
  \end{questionmultx}
  
  \begin{questionmultx}{7375sd}
    73.75 arrondi à 74
    \begin{center}
      \AMCnumericChoices{73.75}{decimals=0,digits=4,sign=false}
    \end{center}
  \end{questionmultx}
  
  \begin{questionmultx}{7375ad}
    73.75
    \begin{center}
      \AMCnumericChoices{73.75}{decimals=2,digits=4,sign=false}
    \end{center}
  \end{questionmultx}
}

\element{Base 2}{
  \begin{questionmultx}{2-73sd}
    73 - base 2 - 1001001
    \begin{center}
      \AMCnumericChoices{73}{decimals=0,digits=12,sign=false,base=2,vertical=true}
    \end{center}
  \end{questionmultx}
  
  \begin{questionmultx}{2-73ad}
    73 - base 2 - 1001001
    \begin{center}
      \AMCnumericChoices{73}{decimals=5,digits=12,sign=false,base=2,vertical=true}
    \end{center}
  \end{questionmultx}
  
  \begin{questionmultx}{2-730sd}
    73.0 - base 2 - 1001001
    \begin{center}
      \AMCnumericChoices{73.0}{decimals=0,digits=12,sign=false,base=2,vertical=true}
    \end{center}
  \end{questionmultx}
  
  \begin{questionmultx}{2-730ad}
    73.0 - base 2 - 1001001
    \begin{center}
      \AMCnumericChoices{73.0}{decimals=5,digits=12,sign=false,base=2,vertical=true}
    \end{center}
  \end{questionmultx}
  
  \begin{questionmultx}{2-7375sd}
    73.75 - base 2 - arrondi à 74 = 1001010
    \begin{center}
      \AMCnumericChoices{73.75}{decimals=0,digits=12,sign=false,base=2,vertical=true}
    \end{center}
  \end{questionmultx}
  
  \begin{questionmultx}{2-7375ad}
    73.75 - base 2 - 1001001.11
    \begin{center}
      \AMCnumericChoices{73.75}{decimals=5,digits=12,sign=false,base=2,vertical=true}
    \end{center}
  \end{questionmultx}
}

\exemplaire{1}{
  \AMCsection{Base 10}
  \restituegroupe{Base 10}
  \AMCsection{Base 2}
  \restituegroupe{Base 2}
}

\end{document}
