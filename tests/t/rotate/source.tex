%
% Copyright (C) 2012-2021 Alexis Bienvenue  <paamc@passoire.fr>
%
% This file is part of Auto-Multiple-Choice
%
% Auto-Multiple-Choice is free software: you can redistribute it
% and/or modify it under the terms of the GNU General Public License
% as published by the Free Software Foundation, either version 2 of
% the License, or (at your option) any later version.
%
% Auto-Multiple-Choice is distributed in the hope that it will be
% useful, but WITHOUT ANY WARRANTY; without even the implied warranty
% of MERCHANTABILITY or FITNESS FOR A PARTICULAR PURPOSE.  See the GNU
% General Public License for more details.
%
% You should have received a copy of the GNU General Public License
% along with Auto-Multiple-Choice.  If not, see
% <http://www.gnu.org/licenses/>.
%
\documentclass[a4paper]{article}

\usepackage[utf8x]{inputenc}    
\usepackage[T1]{fontenc}

\usepackage[francais,bloc,completemulti]{automultiplechoice}    
\geometry{hmargin=5cm,top=6.5cm,bottom=8.5cm}
\begin{document}

\AMCrandomseed{1237893}

\element{general}{
  \begin{question}{mtblanc}
    Quelle est l'altitude du Mont Blanc ?
    \begin{reponses}
      \bonne{4810m}
      \mauvaise{1436m}
      \mauvaise{8848m}
    \end{reponses}
  \end{question}
}

\element{general}{
  \begin{question}{arms}
    En quelle ann\'ee est n\'e Louis Armstrong ?
    \begin{reponses}
      \bonne{1901}
      \mauvaise{1910}
      \mauvaise{1918}
    \end{reponses}
  \end{question}
}

\element{general}{
  \begin{questionmult}{gre}
    Quelles grenouilles avons-nous \'etudi\'e cette ann\'ee ? 
    \begin{reponses}
      \bonne{Les grenouilles-taureau}
      \bonne{Les grenouilles-criquet}
      \mauvaise{Les grenouilles-concombre}
      \mauvaise{Les grenouilles-enclume}
    \end{reponses}
  \end{questionmult}
}

\exemplaire{5}{    

\noindent{\bf QCM  \hfill TEST}
\vspace{5mm}

\champnom{\fbox{    
    \begin{minipage}{.5\linewidth}
      Nom et prénom :
      
      \vspace*{.5cm}\dotfill
      \vspace*{1mm}
    \end{minipage}
  }}
\vspace{5mm}

\melangegroupe{general}
\restituegroupe{general}

}  

\end{document}
