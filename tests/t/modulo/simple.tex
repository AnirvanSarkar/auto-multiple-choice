\documentclass[a4paper]{article}

\usepackage[utf8x]{inputenc}    
\usepackage[T1]{fontenc}

\usepackage[francais,bloc,completemulti]{automultiplechoice}    
\begin{document}

\exemplaire{10}{    

%%% debut de l'en-tête des copies :    

\noindent{\bf QCM  \hfill TEST}

\vspace*{.5cm}
\begin{minipage}{.4\linewidth}
\centering\large\bf Test\\ Examen du 01/01/2008\end{minipage}
\champnom{\fbox{    
                \begin{minipage}{.5\linewidth}
                  Nom et prénom :

                  \vspace*{.5cm}\namefielddots   
                  \vspace*{1mm}
                \end{minipage}
         }}

\vspace{1ex}

%%% fin de l'en-tête

\begin{questionmultx}{mult19}
\scoring{formula="(intX<=20 ? 0 : intX \% 19 == 0 ? 2 : intX \% 19 == 1 ? 1 : intX \% 19 == 18 ? 1 : 0)"}

Un multiple de 19 (différent de 0 et 19) :

\AMCnumericChoices{38}{
  digits=4,
  decimals=0,
  sign=false,
  scoring=false
}

\end{questionmultx}

\begin{questionmultx}{mult7}
\scoring{formula="(intX \% 7 == 0 ? 2 : intX \% 7 == 1 ? 1 : intX \% 7 == 6 ? 1 : 0)"}

Un multiple de 7 :

\AMCnumericChoices{21}{
  digits=2,
  decimals=0,
  sign=false,
  scoring=false
}

\end{questionmultx}

\begin{questionmultx}{mult5}
\scoring{formula="(intX \% 5 == 0 ? 2 : intX \% 5 == 1 ? 1 : intX \% 5 == 4 ? 1 : 0)"}

Un multiple de 5 :

\AMCnumericChoices{15}{
  digits=4,
  decimals=0,
  sign=false,
  scoring=false
}

\end{questionmultx}
\begin{questionmultx}{mult9}
\scoring{formula="(intX \% 2 == 1 && intX \% 9 == 0 ? 2 : intX \% 9 == 0 ? 1 : 0)"}

Un multiple impair de 9 :

\AMCnumericChoices{27}{
  digits=3,
  decimals=0,
  sign=false,
  scoring=false
}

\end{questionmultx}

}   

\end{document}
