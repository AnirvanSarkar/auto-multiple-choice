%
% Copyright (C) 2012-2022 Alexis Bienven�e
%
% This file is part of Auto-Multiple-Choice
%
% Auto-Multiple-Choice is free software: you can redistribute it
% and/or modify it under the terms of the GNU General Public License
% as published by the Free Software Foundation, either version 2 of
% the License, or (at your option) any later version.
%
% Auto-Multiple-Choice is distributed in the hope that it will be
% useful, but WITHOUT ANY WARRANTY; without even the implied warranty
% of MERCHANTABILITY or FITNESS FOR A PARTICULAR PURPOSE.  See the GNU
% General Public License for more details.
%
% You should have received a copy of the GNU General Public License
% along with Auto-Multiple-Choice.  If not, see
% <http://www.gnu.org/licenses/>.
%
\documentclass[a4paper]{article}

\usepackage[latin1]{inputenc}
\usepackage[T1]{fontenc}

\usepackage{multicol}

\usepackage[frenchb]{babel}

\usepackage[bloc,completemulti,calibration,correcindiv]{automultiplechoice}

\begin{document}

\AMCinterIrep=.75ex
\AMCrandomseed{1237893}

\setlength{\multicolsep}{.7ex}

\element{gro}{
  \begin{question}{gro-1}
    G�n�ral/1
    \begin{choiceshoriz}
      \correctchoice{Parfait !}
      \wrongchoice{Pas du tout}
      \wrongchoice{Ce n'est pas �a}
      \wrongchoice{�a non plus}
    \end{choiceshoriz}
  \end{question}
}

\element{gro}{
  \begin{question}{gro-2}
    G�n�ral/2
    \begin{choiceshoriz}
      \correctchoice{Oui}
      \wrongchoice{Erron�e}
      \wrongchoice{Ce n'est pas la bonne}
      \wrongchoice{Ne convient pas}
      \wrongchoice{Non}
    \end{choiceshoriz}
  \end{question}
}

\element{gro}{
  \begin{question}{gro-3}
    G�n�ral/3
    \begin{choiceshoriz}
      \correctchoice{Splendide}
      \wrongchoice{Inutile}
      \wrongchoice{N�gatif}
    \end{choiceshoriz}
  \end{question}
}


%%%%%%%%%%%%%%%%%%%%%%%%%%%%%%%%%%%%%%%%%%%%%%%%%%%%%%%%%%%%%%%%%%%%%%
%%%%%%%%%%%%%%%%%%%%%%%%%%%%%%%%%%%%%%%%%%%%%%%%%%%%%%%%%%%%%%%%%%%%%%

\exemplaire{5}{
\noindent{\bf AMC  \hfill M�lange de rien du tout}

\begin{center}\em
  Exemple pour faire fonctionner les mod�les
\end{center}
\vspace{1ex}


{\setlength{\parindent}{0pt}\hspace*{\fill}\AMCcode{etu}{8}\hspace*{\fill}
\begin{minipage}{5.5cm}
$\longleftarrow{}$\hspace{0pt plus 1cm} codez votre num�ro d'�tu\-diant ci-contre, et �crivez votre nom et pr�nom ci-dessous.

\vspace{3ex}

\champnom{\fbox{    
    \begin{minipage}{.86\linewidth}
      Nom et pr�nom :
      
      \vspace*{.5cm}\dotfill
      
      \vspace*{.5cm}\dotfill
      \vspace*{1mm}
    \end{minipage}
  }}\end{minipage}\hspace*{\fill}
}

\vspace{1ex}

\noindent\hrulefill

\vspace{2ex}

%%%%%%%%%%%%%%%%%%%%%%%%%%%%%%%%%%%%%%%%%%%%%%%%%%%%%%%%%%%%%%%%%%%%%%
%%%%%%%%%%%%%%%%%%%%%%%%%%%%%%%%%%%%%%%%%%%%%%%%%%%%%%%%%%%%%%%%%%%%%%

\shufflegroup{gro}
\insertgroup{gro}


}

\end{document}
